\chapter{Einletung}
Die Programmiersprache Python dominiert, wenn es darum geht, Modelle oder NLP-basierte Systeme zu entwickeln. Python bedankt seine Beliebtheit nicht nur seines relativ einfaches Syntax, sondern auch der großen vielfalt an Bibliotheken für Deep Learning oder NLP. Klassisches Vorgehen ist dabei, zuerst ein Modell mit Daten zu trainieren und dieses in Form von Mikroservices dem Frontend anzubieten. Mobile Geräte und Browser haben aber nicht die Ressourcen, um die immer größer werdenden Modelle handzuhaben. Mit TensorFlow.js hat Google TensorFlow in den Browser und somit in die Welt von JavaScript hineingebracht. Es ist nun möglich, ein bereits trainiertes Modell in TensorFlow.js zu bringen oder ein Modell direkt in JavaScript zu entwickeln. TensorFlow.js ist die erste vollwertige Bibliothek in Industriequalität für die Erstellung neuronaler Netzwerke in JavaScript. 

Sobald ein Modell für maschinelles Lernen trainiert wird, muss es irgendwo bereitgestellt werden, um Vorhersagen über reale Daten zu treffen (\zb Klassifizierung von Bildern und Text, Erkennen von Ereignissen in Audio- oder Videostreams usw.). Ohne Bereitstellung ist das Training eines Modells nur eine Verschwendung von Rechenleistung. Es ist oft wünschenswert oder zwingend erforderlich, dass das \enquote{Irgendwo} ein Web-Frontend ist.

Auf Desktops und Laptops ist der Webbrowser das dominierende Mittel, über die Benutzer auf Inhalte und Dienste im Internet zugreifen. Auf diese Weise verbringen Desktop- und Laptop-Benutzer die meiste Zeit mit diesen Geräten. Auch auf mobilen Geräten, können diese Seiten aufgerufen werden. Auf diese Weise erledigen Benutzer einen Großteil ihrer täglichen Arbeit, bleiben in Verbindung und unterhalten sich. Die breite Palette von Anwendungen, die im Webbrowser ausgeführt werden, bietet umfangreiche Möglichkeiten für die Anwendung clientseitigen maschinellen Lernens. Die browserbasierte Anwendung von Deep Learning bietet zusätzliche Vorteile: geringere Serverkosten, geringere Inferenzlatenz, Datenschutz, sofortige GPU Beschleunigung und sofortiger Zugriff.

Natürlich kann TensorFlow.js, TensorFlow oder große Modelle, die darin entwickelt werden nicht ersetzen. TensorFlow.js kann jedoch viele Probleme lösen und erreicht auch mit kleinen Modellen relativ gute und schelle Ergebnisse.


% \section{Fremdsprachige Begriffe}

% Wenn Sie Ihre Arbeit auf Deutsch verfassen, gehen Sie sparsam mit englischen Ausdrücken um. Natürlich brauchen Sie etablierte englische Fachbegriffe, wie z.\,B. \textit{Interrupt}, nicht zu übersetzen. Sie sollten aber immer dann, wenn es einen gleichwertigen deutschen Begriff gibt, diesem den Vorrang geben. Den englischen Begriff (\textit{term}) können Sie dann in Klammern oder in einer Fußnote\footnote{Englisch: \textit{footnote}.} erwähnen. Absolut unakzeptabel sind deutsch gebeugte englische Wörter oder Kompositionen aus deutschen und englischen Wörtern wie z.\,B. downgeloadet, upgedated, Keydruck oder Beautyzentrum.


% \section{Zitate}

% \subsection{Zitate im Text}

% Wichtig ist das korrekte Zitieren von Quellen, wie es auch von \cite{Kornmeier2011} dargelegt wird. Interessant ist in diesem Zusammenhang auch der Artikel von \cite{Kramer2009}. Häufig werden die Zitate auch in Klammern gesetzt, wie bei \parencite{Kornmeier2011} und mit Seitenzahlen versehen \parencite[S. 22--24]{Kornmeier2011}.

% Bei Webseiten wird auch die URL und das Abrufdatum mit angegeben \parencite{Gao2017}. Wenn die URL nicht korrekt umgebrochen wird, lohnt es sich, an den Parametern \textit{biburl*penalty} in der \texttt{preambel.tex} zu drehen. Kleinere Werte erhöhen die Wahrscheinlichkeit, dass getrennt wird.

% \subsection{Zitierstile}

% Verwenden Sie eine einheitliche und im gesamten Dokument konsequent durchgehaltene Zitierweise\index{Zitierweise}. Es gibt eine ganze Reihe von unterschiedlichen Standards für das Zitieren und den Aufbau eines Literaturverzeichnisses. Sie können entweder mit Fußnoten oder Kurzbelegen im Text arbeiten. Welches Verfahren Sie einsetzen ist Ihnen überlassen, nur müssen Sie es konsequent durchhalten. Stimmen Sie sich im Vorfeld mit Ihrem Betreuer ab -- diese Vorlage unterstützt alle gängigen Zitierweisen.

% In der Informatik ist das Zitieren mit Kurzbelegen\index{Zitat!Kurzbeleg} im Text (Harvard"=Zitierweise) weit verbreitet, wobei für das Literaturverzeichnis häufig die Regeln der \acs{ACM} oder \acs{IEEE} angewandt werden.\footnote{Einen Überblick über viele verschiedene Zitierweisen finden Sie in der \url{http://amath.colorado.edu/documentation/LaTeX/reference/faq/bibstyles.pdf}}

% Am einfachsten ist es, wenn Sie das \verb+\autocite{}+-Kommando verwenden. Bei diesem Kommando können Sie in der Datei \texttt{perambel.tex} festlegen, wie die Zitate generell aussehen sollen, \zb{} ob sie in Fußnoten erfolgen sollen oder nicht. Wollen Sie von dem globalen Zitierstil abweichen, können Sie weiterhin spezielle Kommandos benutzen:

% \begin{itemize}
% 	\item \verb+\autocite{Willberg1999}+: \autocite{Willberg1999}
% 	\item \verb+\cite{Willberg1999}+: \cite{Willberg1999}
% 	\item \verb+\parencite{Willberg1999}+: \parencite{Willberg1999}
% 	\item \verb+\footcite{Willberg1999}+: \footcite{Willberg1999}
% 	\item \verb+\citeauthor{Willberg1999}+: \citeauthor{Willberg1999}
% 	\item \verb+\citeauthor*{Willberg1999}+: \citeauthor*{Willberg1999}
% 	\item \verb+\citetitle{Willberg1999}+: \citetitle{Willberg1999}
% 	\item \verb+\fullcite{Willberg1999}+: \fullcite{Willberg1999}
% \end{itemize}

% Denken Sie daran, dass das Übernehmen einer fremden Textstelle ohne entsprechenden Hinweis auf die Herkunft in wissenschaftlichen Arbeiten nicht akzeptabel ist und dazu führen kann, dass die Arbeit nicht anerkannt wird. Plagiate\index{Plagiat!Bewertung} werden mit mangelhaft (5,0) bewertet und können weitere rechtliche Schritte nach sich ziehen.


% \subsection{Zitieren von Internetquellen}

% Internetquellen\index{Zitat!Internetquellen} sind normalerweise \textit{nicht} zitierfähig. Zum einen, weil sie nicht dauerhaft zur Verfügung stehen und damit für den Leser möglicherweise nicht beschaffbar sind und zum anderen, weil häufig der wissenschaftliche Anspruch fehlt.\footnote{Eine lesenswerte Abhandlung zu diesem Thema findet sich (im Internet) bei \cite{Weber2006}}

% Wenn ausnahmsweise doch eine Internetquelle zitiert werden muss, z.\,B. weil für eine Arbeit dort Informationen zu einem beschriebenen Unternehmen abgerufen wurden, sind folgende Punkte zu beachten:

% \begin{itemize}
% \item Die Webseite ist in ein PDF Dokument zu drucken und im Anhang der Arbeit beizufügen,
% \item das Datum des Abrufs und die URL sind anzugeben,
% \item verwenden Sie Internet"=Seiten ausschließlich zu illustrativen Zwecken (z.\,B. um einen Sachverhalt noch etwas genauer zu erläutern), aber nicht zur Faktenvermittlung (z.\,B. um eine Ihrer Thesen zu belegen).
% \end{itemize}

% Wenn Sie aufgrund der Natur Ihrer Arbeit sehr viele Internetquellen benötigen, dann können Sie diese statt sie auszudrucken auch in elektronischer Form abgeben (CD/DVD). Als Abgabeformat der elektronischen Quellen ist PDF/A\footnote{Bei PDF/A handelt es sich um ein besonders stabile Variante des \ac{PDF}, die von der  \ac{ISO} standardisiert wurde.} vorteilhaft, weil es von allen Formaten die größte Stabilität besitzt.
% Auf der CD/DVD geben Sie bitte auch eine HTML"=Version des Literaturverzeichnisses ab, in der die Online"=Quellen sowie die gespeicherten PDF"=Dateien verlinkt sind.

% Wikipedia\index{Zitat!Wikipedia} stellt einen immensen Wissensfundus dar und enthält zu vielen Themen hervorragende Artikel. Sie müssen sich aber darüber im Klaren sein, dass die Artikel in Wikipedia einem ständigen Wandel unterworfen sind und nicht als Quelle für wissenschaftliche Fakten genutzt werden sollten. Es gelten die allgemeinen Regeln für das Zitieren von Internetquellen. Sollten Sie doch Wikipedia nutzen müssen, verwenden Sie bitte ausschließlich den Perma"=Link\footnote{Sie erhalten den Permalink über die Historie der Seite und einen Klick auf das Datum.}\index{Permalink} zu der Version der Seite, die Sie aufgerufen haben.


% % Jedes Kapitel besteht aus Unterkapiteln (section)
% \section{Gliederung: Zweite Ebene}

% Die Gliederung im Inhaltsverzeichnis erfolgt mit Kapiteln \verb+\chapter{Titel}+, Abschnitten \verb+\section{Titel}+, Unterabschnitten \verb+\subsection{Titel}+. Zusätzlich können noch Unterunterabschnitte \verb+\subsubsection{Titel}+ und Absätze \verb+\paragraph{Titel}+ verwendet werden. Damit kommt man auf maximal fünf Ebenen, was für eine Abschlussarbeit mehr als ausreichend ist.

% Auf jeder Ebene sollten Sie erläutern, was in den darunter liegenden Ebene beschrieben wird, sodass im Normalfall keine Gliederungsebene leer ist und nur aus Untereinheiten besteht. Im folgenden zeigt dieses Template, wie man weitere Ebenen mit \LaTeX erzeugt.

% % Unterkapitel können noch einmal durch subsections untergliedert
% % werden (jetzt auf der 3. Ebene)
% \subsection{Gliederung: Dritte Ebene}

% % Mit Labels können Sie sich später im Text wieder auf diese Stelle beziehen
% \label{Gliederung:EbeneDrei}

% % Einträge für den Index anlegen. Ein Index wird normalerweise in einer Abschluss-
% % arbeit nicht benötigt.
% \index{Gliederung!Ebenen}

% % Auf der 4. Ebene liegen die subsubsections. In diesem Template bekommt die
% % 4. Ebene keinen Nummern mehr und erscheint auch nicht im Inhaltsverzeichnis.
% \subsubsection{Gliederung: Vierte Ebene}

% % Auf der 5. Ebene werden einzelne Absätze mit Überschriften versehen.
% \paragraph{Gliederung: Fünfte Ebene} Anders als in diesem Beispiel, darf in Ihrer Arbeit kein Gliederungspunkt auf seiner Ebene alleine stehen. D.\,h. wenn es ein 1.1 gibt, muss es auch ein 1.2 geben.
